%\renewcommand{\headheight}{15pt}
%%\renewcommand{\headsep}{0pt}
%%\renewcommand{\topmargin}{-20pt}
%%\renewcommand{\textwidth}{110mm}
%\renewcommand{\textheight}{165mm}
%\addtolength{\textwidth}{6mm}
%%\addtolength{\voffset}{-22mm}
%\addtolength{\topmargin}{-22mm}
%%\addtolength{\hoffset}{-3mm}
%\addtolength{\evensidemargin}{-3mm}
%\addtolength{\oddsidemargin}{-3mm}


%\newcommand{\copyrightinput}[1]{\input copyright/#1}

\newcommand{\copyrightinput}[1]{\input copyright/#1}
\newcommand{\unlikely}[1]{\input copyright/#1}
\newcommand{\permissioned}[1]{\input copyright/#1}
\newcommand{\FreeAlternative}[1]{}

% For setting the gap between verse numbers and verses:
\setlength{\vleftskip}{1em}

%\includeonly{devotions}

\graphicspath{{./}{./images/}{./motets/}{./chant/}{./rounds/}}

\DeclareGraphicsExtensions{.pdf}

%\renewcommand*\oldstylenums[1]{{\fontfamily{fxlj}\selectfont #1}}

\def\atitle#1{{\centering %\textbf
                       {\Large \scshape #1}\par}}

\def\heading#1{{\centering \textsc{\large #1}\par}}

\def\smallhead#1{{\centering \textit{#1}\par}}

%\def\rubrics#1{{\centering \textcolor{red}{#1} \par}}
\def\rubrics#1{{\centering \emph{#1} \par}}


\newcommand{\sideResponse}[2]{\makebox[\hsize]{#1 \hfill%
            \begin{rotate}{-90}\textit{#2}\end{rotate}\hskip6pt}}


%\newcommand{\westThumb}[2]{}%\addthumb{#1}{\footnotesize\textbf{#2}}{white}{black}}


%\newcommand\sottovoce{\footnotesize}
\newenvironment{sottovoce}{\begin{multicols}{2}\footnotesize}%
{\end{multicols}}

\newenvironment{indentedVerse}[1][2]{%
              \addtolength{\leftmargin}{3em}
	      \addtolength{\vleftskip}{-1em}}%
       {\addtolength{\leftmargin}{-3em}%
       \addtolength{\vleftskip}{1em}}


\setlength{\parskip}{6pt plus 2pt minus 1pt}
\setlength{\parindent}{0pt}
\setlength{\stanzaskip}{6pt plus 2pt minus 1pt}

%\renewcommand{\sectionmark}[1]{\markboth{#1}{}}
%\renewcommand{\subsectionmark}[1]{\markboth{#1}{}}

% \pagestyle{empty}
% \fancyhead{}
% \fancyfoot{}
% \fancyhead[LE,RO]{\thepage}
% \fancyhead[EC]{\scshape \nouppercase \leftmark}
% \fancyhead[OC]{\scshape \rightmark}

% \renewcommand{\headrulewidth}{0pt}
% \renewcommand{\footrulewidth}{0pt}

\newcounter{hymnNo}
%\newcounter{numVerbum}
%\newcounter{numTantum}
%\newcounter{numReceiveOFather}
%\newcounter{numTemp}



%
% cludgy way to handle old/new style numerals
%

%\newcommand\upnums[1]{{\fontfamily{fxl}\selectfont #1}}
%\newcommand\upnums[1]{{#1}}
%\newcommand\upnums[1]{\textl{#1}}
\newcommand\upnums[1]{{\addfontfeature{Numbers=Lining} #1}}

%
% Memoir redefines the index package
%

%\makeindex
%\newindex[thehymnNo]{aut}{adx}{and}{Index of Authors}
%\renewindex[thehymnNo]{default}{idx}{ind}{Index of First Lines}



\newcommand\sidenote[1]{}

\def\marginote#1{\marginpar{\textsf{%\footnotesize 
                                      \scshape #1}}}

\def\stand{\marginote{\vskip2pt Stand}}
\def\sit{\marginote{\quad Sit}}
\def\kneel{\marginote{Kneel}}
\def\HMstand{\marginote{Sung\vskip-3pt Mass\vskip-3pt Stand}}
\def\ringbell{\marginote{\includegraphics[width=5mm]{bell}}}
\def\threebell{\marginote{\includegraphics[width=5mm]{bell}\includegraphics[width=5mm]{bell}\includegraphics[width=5mm]{bell}}}


% my First Line
% sets the width of the verse
% indexes the first line

\newcommand\FirstLine[1]{%\stepcounter{hymnNo}%
    \firstlindex{#1}%
    \settowidth{\versewidth}{#1}}

\newcommand\newHymn{\refstepcounter{hymnNo}}

\newcommand\fortranC[1]{{\rightline{\footnotesize{For translation, see
#1}}}}
\newcommand\fororigC[1]{{\rightline{\footnotesize{For original, see
#1}}}}

%\newcommand\pointtrans{\marginpar{\footnotesize Translation follows.}}%
\newcommand\pointtrans{\marginpar{\raggedright\footnotesize For English see below.}}

%\newcommand\pointorig{\marginpar{\footnotesize Original precedes.}}%
\newcommand\pointorig{\marginpar{\raggedright\footnotesize For original see above.}}


%\newcommand\fortranC[1]{{\small \centering [\textit{For translation, see #1}]}}

%From the verse package:

\renewcommand{\poemtitlefont}{\normalfont\large\itshape\centering}
\newcommand{\attrib}[1]{%
   \nopagebreak{\raggedleft\footnotesize #1\par}}


%
%
%  Putting all the initials together
%
%
%\renewcommand{\LettrineFontHook}{\fontencoding{T1}\color{red}\fontfamily{anavio}}
\renewcommand{\LettrineFontHook}{\anavio}


% my First Verse sets the first word in small caps with a bold initial
% plus puts the hymn number in the left margin

%\newcommand\initialV[2]{{\LARGE \scshape {\fontencoding{T1}\fontfamily{anavio}\selectfont\color{red}{#1}}\normalsize{#2}}}%
\newcommand\initialV[2]{{\LARGE {%\fontencoding{T1}\fontfamily{anavio}\selectfont
\anavio\textbf{#1}}\normalsize\textsc{#2}}}%

\newcommand\FirstVerse[2]{%\stepcounter{hymnNo}%
                 \hskip-\leftmargin\hskip\vindent%
%		 \rlap{%\fontfamily{fxl}\selectfont 
                 \makebox[0cm][l]{%
		        \bfseries \huge\upnums{\thehymnNo}}%
		 \hskip\leftmargin\hskip-\vindent\ignorespaces%
		 {\initialV{#1}{#2}}}

% my Hymn Title sets the title in small caps
% and indexes the title
% plus puts the hymn number in the left margin
% for litanies

\newcommand\HymnTitlesc[1]{%\stepcounter{hymnNo}%
%   \specialindex{firstlines}{hymnNo}{#1|upnums}%
    \firstlindex{#1}%
   \makebox[0cm][l]{%\fontfamily{fxl}\selectfont 
                 \bfseries \huge\upnums{\thehymnNo}}%
   \centerline{ \scshape #1} \vskip-5pt}

\newcommand\HymnTitleit[1]{%\stepcounter{hymnNo}%
%   \specialindex{firstlines}{hymnNo}{#1|upnums}%
    \firstlindex{#1}%
   \makebox[0cm][l]{%\fontfamily{fxl}\selectfont 
                 \bfseries \huge\upnums{\thehymnNo}}%
   \centerline{ \itshape #1} \vskip-5pt}

\newcommand\HymnTitleInVerse[1]{%
%   \specialindex{firstlines}{hymnNo}{#1|upnums}%
    \firstlindex{#1}%
   \hskip-\leftmargin\hskip\vindent%
%   \rlap{%\fontfamily{fxl}\selectfont 
   \makebox[0cm][l]{%
                  \bfseries \huge\upnums{\thehymnNo}}%
   \hskip\leftmargin\hskip-\vindent\ignorespaces%
   \makebox[\versewidth][c]{\itshape #1}\par }

\newcommand\JustHymnNum{{%\fontfamily{fxl}\selectfont 
\bfseries \huge\upnums{\thehymnNo}}\par\vskip-5pt}

\newcommand\JustHymnNumInVerse{\hskip-\leftmargin\hskip\vindent%
%		 \rlap{%\fontfamily{fxl}\selectfont 
                 \makebox[0cm][l]{%
		        \bfseries \huge\upnums{\thehymnNo}}%
		 \hskip\leftmargin\hskip-\vindent\ignorespaces}

\newcommand\Htitle[1]{\specialindex{sources}{hymnNo}{#1}}
\newcommand\Hpoet[2]{\specialindex{sources}{hymnNo}{#1}\rightline{\footnotesize{#1, #2}}}
\newcommand\Htrans[2]{\specialindex{sources}{hymnNo}{#1}\rightline{\footnotesize{Tr. #1, #2}}}
\newcommand\Hmeter[1]{}%{\footnotesize #1}}
\newcommand\Hsource[2]{}%{\footnotesize #1, #2}}
\newcommand\Hnote[1]{\rightline{\footnotesize{#1}}}
\newcommand\Hlnote[1]{{\raggedleft\footnotesize #1\par}}
\newcommand\Hcopyright[2]{}
\newcommand\Htune[1]{\rightline{\footnotesize{#1}}}

%\renewcommand\source[1]{\rightline{\footnotesize #1}}
%\setcounter{secnumdepth}{-1}
\newlength{\gcolwidth}
\setlength{\gcolwidth}{0.48\textwidth}

\def\latin#1{%
\selectlanguage{latin}
\ParallelLText{#1\strut}%
\selectlanguage{british}%
\relax %
}

\def\vern#1{%
\ParallelRText{#1\strut}%
%\ParallelLText{\ \vspace*{1mm}}\ParallelRText{\ \vspace*{1mm}}%
\ParallelPar %
%\kern -1mm%
\relax %
}

\def\firstlatin#1#2#3{%
\latin{%
\noindent\begin{minipage}[t]{\ParallelLWidth}%
\lettrine{#1}{#2}#3\strut%\vspace{0.9mm}\vspace{0.13cm}%
\end{minipage}%
}%
%\vspace{-0.2cm}%
}

\def\firstvern#1#2#3{%
\vern{%
\noindent\begin{minipage}[t]{\ParallelRWidth}%
\lettrine{#1}{#2}#3\strut%\vspace{0.9mm}\vspace{0.13cm}%
\end{minipage}%
}%\vspace{-0.2cm}%
}

\def\firstlatincludge#1#2#3{%
\latin{%
\noindent\begin{minipage}[b]{\ParallelLWidth}%
\lettrine{#1}{#2}#3\strut%\vspace{0.9mm}\vspace{0.13cm}%
\end{minipage}%
}%
%\vspace{-0.2cm}%
}

\def\firstverncludge#1#2#3{%
\vern{%
\noindent\begin{minipage}[b]{\ParallelRWidth}%
\lettrine{#1}{#2}#3\strut%\vspace{0.9mm}\vspace{0.13cm}%
\end{minipage}%
}%\vspace{-0.2cm}%
}

\newdimen\temp 
\def\twocolspace#1{%
\temp=#1%
\advance\temp by 1mm%
\latin{\ \vspace{\temp}}\vern{\ \vspace{\temp}}%
\kern -1mm%
\relax %
}

\def\s{\relax}

\newcommand\source[1]{\rightline{\footnotesize #1}}

%\newlength\ay
%\newlength\bee

%\newcommand\lateng[3]{\setlength{\ay}{0.48\columnwidth}
%                      \setlength{\bee}{0.48\columnwidth}
%                      \advance\ay by -#1
%                      \advance\bee by #1
%                      \noindent\begin{minipage}[t]{\ay}#2\strut
%                      \end{minipage}\hfill{}
%                      \noindent\begin{minipage}[t]{\bee}#3\strut
%                      \end{minipage}{\smallskip}}

%\renewcommand\rubrics[1]{\noindent\textit{#1}}

%\newcommand\sidenote[1]{}

%\newenvironment{scoles}
%		{\onehalfspacing
%		\setlength{\parskip}{\smallskipamount}
%		\setlength{\parindent}{0pt}}
%		{\singlespacing}

%\def\hymn#1{\subsection*{#1}\index{#1}}

\newlength{\latengoffset}

%\newenvironment{psalmmode}
%	       {\begin{list}{}{%
%		     \setlength{\itemindent}{-2em}
%		     \setlength{\listparindent}{-2em}
%		     \setlength{\parsep}{1pt}
%		     \setlength{\topsep}{0pt}
%%		     \setlength{\rightmargin}{\leftmargin}
%		   }
%		   \item[]\ignorespaces}
%	       {\unskip\end{list}}


%
% New wonderful Rbar and Vbar
%
% Using Paul Lloyd's Anavio font
%
% To blend with 11pt Romande
%

%\newfontface{\elevenVR}[Size=11]{Linux Libertine Capitals O Semibold}

\renewcommand\Vbar{\makebox[1em][l]{\versicles v}}
\renewcommand\Rbar{\makebox[1em][l]{\versicles r}}
\newcommand\Pnobar{\makebox[1em][l]{\textsc{p.}}}
\renewcommand\Abar{\makebox[1em][l]{\versicles a}}

\newcommand\maltese{{\tiny \cruci V}\ }

%\font\gregoriansymbolfont=gresym at 12pt

%\def\grestar{% the other (multimulti-pointed) star is char 69. 
%{\gregoriansymbolfont \char 71}%71
%\relax %
%}
%\font\textgregoriansymbolfont = gresym at 12 pt

%\def\textgrestar{% the other (multimulti-pointed) star is char 69. 
%{\textgregoriansymbolfont \char 71}%71
%\relax %
%}

\newbox\linkbox
\newbox\wordbox
%\newdimen\boxwid

\def\lyrlink{%
  % Bogen erstellen: Put the smilie in the box:
  \setbox\linkbox=\hbox{$\smile$}%
  % In Box der Breite eines Wortzwischenraums einsetzen:
  % Make the box one space wide:
  \setbox\linkbox=\hbox to\the\fontdimen2\the\font{%
    \hss
    % Unter die Grundlinie druecken:
    % Lower the box under the word:
    \lower\ht\linkbox\hbox{%
      % Zusaetzlicher vertikaler Abstand zur Wortunterseite:
      % lower the smile a little bit more:
      \lower1pt\hbox{%
	\relax
	  \hbox{$\smile$}%
	}}%
    \hss}%
  % Keine zusaetzliche Tiefe fuer Bogen anrechnen:
  % Make the depth of the box zero:
  \dp\linkbox=0pt
  % Bogen setzen:
  \box\linkbox}

\def\intralink#1{%
  \setbox\linkbox=\hbox{$\smile$}%
  \setbox\wordbox=\hbox{#1}%
  \setbox\linkbox=\hbox to\wd\wordbox{%
    \hss
    % Unter die Grundlinie druecken:
    % Lower the box under the word:
    \lower\ht\linkbox\hbox{%
      % Zusaetzlicher vertikaler Abstand zur Wortunterseite:
      % lower the smile a little bit more:
      \lower1pt\hbox{%
	\relax
	  \hbox{$\smile$}%
	}}%
    \hss}%
  % Keine zusaetzliche Tiefe fuer Bogen anrechnen:
  % Make the depth of the box zero:
  \dp\linkbox=0pt
%  \let\boxwid=\the\wd\wordbox
%  \lower\ht\linkbox%
  \box\wordbox \llap{\box\linkbox}}

\newsavebox{\annotinbox}
\newlength{\annotheight}

\newcommand\annotinverse[1]{\savebox{\annotinbox}{\parbox{5cm}{\raggedleft\scriptsize #1 }}%
\vskip-\baselineskip%
\settoheight{\annotheight}{\usebox{\annotinbox}}%
\addtolength{\annotheight}{\baselineskip}%
\hskip-\leftmargin\hskip\vindent\raisebox{-\annotheight}[0pt][0pt]{\makebox[\hsize]{\hfill \usebox{\annotinbox}}}}

\newenvironment{psalmmode}
	       {\vskip-\baselineskip\begin{list}{}{%
		     \setlength{\itemindent}{-2em}
		     \setlength{\listparindent}{-2em}
		     \setlength{\parsep}{1pt}
		     \setlength{\topsep}{0pt}
%		     \setlength{\rightmargin}{\leftmargin}
		   }
		   \item[]\ignorespaces}
	       {\unskip\end{list}}


