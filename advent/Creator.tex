\newHymn
\FirstLine{Creator of the starry skies}

\settowidth{\versewidth}{An ear to Thy poor suppliants give}

\begin{verse}[\versewidth]
\begin{altverse}
\FirstVerse{C}{reator} of the starry skies!\\*
Eternal Light of all who live!\\*
Jesu, Redeemer of mankind!\\*
An ear to Thy poor suppliants give.
\end{altverse}
\pointorig

\begin{altverse}
\flagverse{2} When man was sunk in sin and death,\\*
Lost in the depth of Satan's snare,\\*
Love brought Thee down to cure our ills,\\*
By taking of those ills a share.
\end{altverse}

\begin{altverse}
\flagverse{3} Thou, for the sake of guilty men,\\*
Causing Thine own pure blood to flow,\\*
Didst issue from Thy virgin shrine\\*
And to the Cross a Victim go.
\end{altverse}

\begin{altverse}
\flagverse{4} So great the glory of Thy might,\\*
If we but chance Thy name to sound\\*
At once all heaven and hell unite\\*
In bending low with awe profound.
\end{altverse}

\begin{altverse}
\flagverse{5} Great Judge of all! in that last day\\*
When friends shall fail and foes combine,\\
Be present then with us, we pray,\\*
To guard us with Thy arm divine.
\end{altverse}

\begin{altverse}
\flagverse{6}To God the Father, and the Son,\\*
All praise and power and glory be;\\
With Thee, O holy Comforter!\\*
Henceforth through all eternity.
\end{altverse}


%6. Power, honour, praise, and glory,\\*  %%% Where on earth did these
% words come from????
%To God the Father and the Son,\\*
%And also to the holy Paraclete,\\*
%While eternal ages run.

\end{verse}

%\Hpoet{}{}

\Htrans{Edward Caswall}{1814--78}

%The hymn Conditor alme siderum was an anonymous text from the 7th century (trans- "Creator of the stars of night") Under Pope
%Urban VIII, the Roman Breviary was revised in 1632, and this hymn was greatly altered, becoming Creator alme siderum. The hymn
%in the Liber usualis is Creator alme siderum, which maintains classic Latin poetry meters. The text translates, "Bright builder
%of the heavenly poles."



